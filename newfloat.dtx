% \iffalse meta-comment
%<=*COPYRIGHT>
%% Copyright (c) 2011-2022 by Martin Scharrer <martin.scharrer@web.de>
%% ---------------------------------------------------------------------------
%% This work may be distributed and/or modified under the
%% conditions of the LaTeX Project Public License, either version 1.3
%% of this license or (at your option) any later version.
%% The latest version of this license is in
%%   http://www.latex-project.org/lppl.txt
%% and version 1.3 or later is part of all distributions of LaTeX
%% version 2005/12/01 or later.
%%
%% This work has the LPPL maintenance status `maintained'.
%%
%% The Current Maintainer of this work is Martin Scharrer.
%%
%% This work consists of the files newfloat.dtx and newfloat.ins
%% and the derived filebase newfloat.sty.
%%
%<=/COPYRIGHT>
% \fi
%
% \iffalse
%<*driver>
\ProvidesFile{newfloat.dtx}[%
%<=*DATE>
    2011/10/22
%<=/DATE>
%<=*VERSION>
    v1.0
%<=/VERSION>
    Define new float environments the LaTeX core way]
\documentclass{ydoc}
\GetFileInfo{newfloat.dtx}
\usepackage{newfloat}[\filedate]
\EnableCrossrefs
\CodelineIndex
\RecordChanges
\begin{document}
  \DocInput{\jobname.dtx}
  \PrintChanges
  \PrintIndex
\end{document}
%</driver>
% \fi
%
% \CheckSum{168}
%
% \CharacterTable
%  {Upper-case    \A\B\C\D\E\F\G\H\I\J\K\L\M\N\O\P\Q\R\S\T\U\V\W\X\Y\Z
%   Lower-case    \a\b\c\d\e\f\g\h\i\j\k\l\m\n\o\p\q\r\s\t\u\v\w\x\y\z
%   Digits        \0\1\2\3\4\5\6\7\8\9
%   Exclamation   \!     Double quote  \"     Hash (number) \#
%   Dollar        \$     Percent       \%     Ampersand     \&
%   Acute accent  \'     Left paren    \(     Right paren   \)
%   Asterisk      \*     Plus          \+     Comma         \,
%   Minus         \-     Point         \.     Solidus       \/
%   Colon         \:     Semicolon     \;     Less than     \<
%   Equals        \=     Greater than  \>     Question mark \?
%   Commercial at \@     Left bracket  \[     Backslash     \\
%   Right bracket \]     Circumflex    \^     Underscore    \_
%   Grave accent  \`     Left brace    \{     Vertical bar  \|
%   Right brace   \}     Tilde         \~}
%
%
% \changes{v1.0}{2011/10/22}{First release.}
%
% \DoNotIndex{\newcommand,\newenvironment}
%
% \GetFileInfo{newfloat.dtx}
% \author{Martin Scharrer}
% \email{martin.scharrer@web.de}
% \ifdefined\repository
%   \repository{https://github.com/MartinScharrer/newfloat/}
% \fi
%
% \maketitle
%
% \begin{abstract}\noindent
% This package allows to define new float environment using
% the exact same way as the \LaTeX\ core floats \env{figure}
% and \env{table}.
% \end{abstract}
%
% \section{Introduction}
% Often it is useful to have custom float environments with own captions.
% There are already some ways available to define new floats, like the \pkg{float} and \pkg{caption} packages
% or the KOMA-Script and \cls{memoir} classes.
% However, some of them, like \pkg{float}, use custom styles which do not allow for arbitrary formatting of the environment
% content. While the caption position can be chosen directly for the core floats by placing one or more \cs{caption} macros accordantly, 
% this is not possible with floats defined by the \pkg{float} package. The KOMA-Script classes and \pkg{caption} allow for such free-style floats, but
% are rather large.
%
% This package provides a simple interface to define new free-style floats using the same core macros used for \env{figure} and \env{table}.
% The functionality is limited to this by design. Users which want to have more functionality should try the above mentioned packages and classes.
%
% \section{Usage}
% After loading the package new floats can be defined using the following macro. It should only be used in the document preamble.
%
% \DescribeMacro\Newfloat[<name for captions>]{<name>}{<extension>}
% This defines a new float environment with the given \meta{name} as well as \env{\meta{name}*} for use with |twocolumn| mode to place it over both columns.
% A file named \Macro\jobname'.'<extension> will be used to store the list of floats of this type if either \cs{listof<name>s} or \Macro\Listof{<name>} was used.
% A \meta{name for captions} can be given using the optional argument, otherwise the normal \meta{name} is used with the first letter capitalized.
%
% \DescribeMacro\Listof{<name>}
% Prints the \emph{List of \meta{name}} listing like e.g.\ \Macro\listoffigures does for figures.
% Alternatively a \cs{listof<name>s} can be used, i.e.\ \Macro\Listof{foobar} is identical to \Macro\listoffoobars.
%
% \section{Compatibility}
% This package is compatible with the \LaTeX\ standard classes which code it uses internally.
% The macro name \Macro\Newfloat was chosen instead of \Macro\newfloat to be compatible with the \pkg{float} package which define this macro.
% The compatibility beyond this is not tested yet.
%
% \StopEventually{}
% \clearpage
% \section{Implementation}
%
% \iffalse
%<@newfloat.sty>
% \fi
%
% \Finale
\endinput
